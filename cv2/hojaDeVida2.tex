%%%%%%%%%%%%%%%%%%%%%%%%%%%%%%%%%%%%%%%%%
% Developer CV
% LaTeX Template
% Version 1.0 (28/1/19)
%
% This template originates from:
% http://www.LaTeXTemplates.com
%
% Authors:
% Jan Vorisek (jan@vorisek.me)
% Based on a template by Jan Küster (info@jankuester.com)
% Modified for LaTeX Templates by Vel (vel@LaTeXTemplates.com)
%
% License:
% The MIT License (see included LICENSE file)
%
%%%%%%%%%%%%%%%%%%%%%%%%%%%%%%%%%%%%%%%%%

%----------------------------------------------------------------------------------------
%	PACKAGES AND OTHER DOCUMENT CONFIGURATIONS
%----------------------------------------------------------------------------------------

\documentclass[9pt]{developercv} % Default font size, values from 8-12pt are recommended

%----------------------------------------------------------------------------------------

\begin{document}

%----------------------------------------------------------------------------------------
%	TITLE AND CONTACT INFORMATION
%----------------------------------------------------------------------------------------

\begin{minipage}[t]{0.45\textwidth} % 45% of the page width for name
	\vspace{-\baselineskip} % Required for vertically aligning minipages
	
	% If your name is very short, use just one of the lines below
	% If your name is very long, reduce the font size or make the minipage wider and reduce the others proportionately
	\colorbox{black}{{\HUGE\textcolor{white}{\textbf{{Hildebrando}}}}} % First name
	
	\colorbox{black}{{\HUGE\textcolor{white}{\textbf{\MakeUppercase{Vargas B.}}}}} % Last name
	
	\vspace{6pt}
	
	{\huge } % Career or current job title
\end{minipage}
\begin{minipage}[t]{0.275\textwidth} % 27.5% of the page width for the first row of icons
	\vspace{-\baselineskip} % Required for vertically aligning minipages
	
	% The first parameter is the FontAwesome icon name, the second is the box size and the third is the text
	% Other icons can be found by referring to fontawesome.pdf (supplied with the template) and using the word after \fa in the command for the icon you want
	\icon{MapMarker}{12}{Med Antioquia}\\
	\icon{Phone}{12}{+57 3059427297}\\
	\icon{At}{12}{\href{mailto:gluoneros@gmail.com}{gluoneros@gmail.com}}\\	
\end{minipage}
\begin{minipage}[t]{0.275\textwidth} % 27.5% of the page width for the second row of icons
	\vspace{-\baselineskip} % Required for vertically aligning minipages
	
	% The first parameter is the FontAwesome icon name, the second is the box size and the third is the text
	% Other icons can be found by referring to fontawesome.pdf (supplied with the template) and using the word after \fa in the command for the icon you want
	\icon{Globe}{12}{\href{https://gluoneros.wordpress.com/}{gluoneros.com}}\\
	\icon{Github}{12}{\href{https://github.com/gluoneros}{github.com/gluoneros}}\\
	\icon{Twitter}{12}{\href{https://twitter.com/@gluoneros}{@gluoneros}}\\
\end{minipage}

\vspace{0.5cm}

%----------------------------------------------------------------------------------------
%	INTRODUCTION, SKILLS AND TECHNOLOGIES
%----------------------------------------------------------------------------------------

\cvsect{Sobre mí?}

\begin{minipage}[t]{0.4\textwidth} % 40% of the page width for the introduction text
	\vspace{-\baselineskip} % Required for vertically aligning minipages
Profesional en Ingeniería de Software y lic en matemáticas y física; versátil ingeniero devops; certificado administrador linux con  6 años de  experiencia. Competente en temas devops y con una sólida formación en administración de sistemas y redes, así como en scripting y automatización con Python y Bash, Con el objetivo de optimizar los procesos de desarrollo y despliegue para facilitar la entrega continua de software de alta calidad.
\end{minipage}
\hfill % Whitespace between
\begin{minipage}[t]{0.5\textwidth} % 50% of the page for the skills bar chart
	\vspace{-\baselineskip} % Required for vertically aligning minipages
	\begin{barchart}{5.5}
		\baritem{Python}{70}
		\baritem{Java}{60}
		\baritem{Servidores linux}{80}
		\baritem{Docker}{55}
		\baritem{Git}{40}
		\baritem{Kubernetes}{60}
		\baritem{Jenkins}{60}
	\end{barchart}
\end{minipage}

\begin{center}
	\bubbles{4/AWS, 2/Azure, 3/google cloud, 4/Django,  3/Flask, 2/FastAPI}
\end{center}

%----------------------------------------------------------------------------------------
%	EXPERIENCE
%----------------------------------------------------------------------------------------

\cvsect{Experiencia}

\begin{entrylist}
	\entry
		{2021 -- Actualidad}
		{Freelance. developer}
		{Freelance.}
		{Freelancer en proyectos python con  \texttt{Python}\slashsep\texttt{Postgres}				   \slashsep\texttt{Django}}
	\entry
		{2010 -- 2021}
		{Docente Matemáticas y física}
		{Seduca.}
		{Docente Matemáticas y física en media académica. Docente de cátedra de física Udea \texttt{Física}\slashsep\texttt{Matemáticas}\slashsep\texttt{Linux}}
	
\end{entrylist}

%----------------------------------------------------------------------------------------
%	EDUCATION
%----------------------------------------------------------------------------------------

\cvsect{Education}

\begin{entrylist}
	\entry
		{2018 -- 2023}
		{Ingeniero de Software}
		{Politécnico Grancolomiano}
		{Comprensión de los principios fundamentales de la ingeniería de software especialmente el desarrollo backend trabajando en la creación de soluciones, utilizando dos de los lenguajes más influyentes en la industria: Python y Java.}
	\entry
		{2006 -- 2009}
		{Licenciado Matemáticas y Física}
		{Universidad de Antioquia}
		{Conocimientos y habilidades y necesarias para la docencia de la física y de las matemáticas en educación media y , así como la universidad.  }
	\entry
		{2004 -- 2006}
		{Ingeniero electrónico}
		{Universidad de Antioquia }
		{Sexto semestre de las áreas, Conocimientos en IoT}
\end{entrylist}

%----------------------------------------------------------------------------------------
%	ADDITIONAL INFORMATION
%----------------------------------------------------------------------------------------

\begin{minipage}[t]{0.4\textwidth}
	\vspace{-\baselineskip} % Required for vertically aligning minipages

	\cvsect{Lenguajes}
	
	\textbf{Spanish} - nativo\\	
	\textbf{English} - proficient, B2\\
	
\end{minipage}
\hfill
\begin{minipage}[t]{0.6\textwidth}
	\vspace{-\baselineskip} % Required for vertically aligning minipages
	
	\cvsect{Proyectos}
	
	\textbf{Ventas en tienda} - Con datos de ventas de una tienda, se limpian los datos y se realiza un extenso análisis de series temporales. \\	
	
	\textbf{SportsPredictor} - Con el uso de estadísticas deportivas se predice qué jugador juvenil tendría la mejor carrera. \\
	
	Desarrollar un modelo de análisis basado en inteligencia artificial para identificar y cuantificar las desigualdades energéticas en zonas rurales y urbanas de Colombia. Evaluando el impacto de estas desigualdades en el desarrollo sostenible y proponer soluciones tecnológicas y políticas para garantizar el acceso equitativo a la energía, mejorando la calidad de vida de las poblaciones más vulnerables.
	
	Indagar las fuentes y cantidades de energía consumidas en sectores tanto urbano como rurales
	
\end{minipage}
\hfill

%----------------------------------------------------------------------------------------

\end{document}
