\documentclass[a4paper,10pt]{article}
\usepackage[margin=1in]{geometry}
\usepackage[utf8]{inputenc}
\usepackage[T1]{fontenc}
\usepackage{enumitem}
\usepackage{titlesec}
\usepackage{hyperref}
\usepackage{xcolor}

\setlength{\parindent}{0pt}
% Colors and links
\definecolor{blue1}{HTML}{1F4E79}
\hypersetup{
	colorlinks=true,
	urlcolor=blue1,
	linkcolor=blue1
}

\titleformat{\section}{\large\bfseries\color{blue1}}{}{0em}{}
\setlist[itemize]{noitemsep, topsep=0pt, leftmargin=1.5em}

\renewcommand{\familydefault}{\sfdefault}

\begin{document}
	
	% Header
	\begin{center}
		{\Huge \textbf{Hildebrando Vargas Bedoya}} \\
		\vspace{4pt}
		Medellín, Colombia \quad | \quad
		\href{mailto:gluoneros@gmail.com}{gluoneros@gmail.com} \quad | \quad
		+57 305 942 7297 \\
		\href{https://linkedin.com/in/hildebrando-vargas}{linkedin.com/in/hildebrando-vargas} \quad | \quad
		\href{https://github.com/gluoneros}{github.com/gluoneros} \quad | \quad
		\href{https://gluoneros.com}{gluoneros.com}
	\end{center}
	
	\vspace{0.5em}
	
	% Perfil Profesional
	\section*{Perfil Profesional}
	Ingeniero de Software con sólida formación en desarrollo backend y más de 10 años de experiencia en educación técnica. Versátil y autodidacta, especializado en el desarrollo de soluciones con Python, Java, y tecnologías cloud. Con experiencia práctica como freelancer, combinando conocimientos en DevOps, automatización de procesos, machine learning y sistemas distribuidos para crear software escalable y eficiente.
	
	% Experiencia
	\section*{Experiencia Profesional}
	\textbf{Freelancer} \hfill 2021 -- Presente \\
	Desarrollo de proyectos de software como backend developer. 
	\begin{itemize}
		\item Implementación de APIs REST con Django, Flask y FastAPI.
		\item Automatización de procesos con Bash y Python.
		\item Despliegue de apps usando Docker, AWS EC2, RDS.
	\end{itemize}
	
	\textbf{Docente de Matemáticas y Física} \hfill 2010 -- 2021 \\
	Universidad de Antioquia y secundaria pública. 
	\begin{itemize}
		\item Diseño de material educativo, evaluación y seguimiento de aprendizaje.
		\item Formación en física, matemáticas y fundamentos de programación.
	\end{itemize}
	
	% Proyectos
	\section*{Proyectos Destacados}
	\textbf{GreenPredEnergy} \hfill Python, Flask, Scikit-learn \\
	ML para predecir viabilidad de energía solar/eólica en municipios de Colombia.
	
	\textbf{Sistema CRUD Inventario} \hfill Flask, MySQL, Docker \\
	App para gestión de inventario con login, backend robusto y ORM con SQLAlchemy.
	
	\textbf{Gestión de Tareas con Java} \hfill Spring Boot, Thymeleaf, Tailwind CSS \\
	Sistema de asignación de tareas entre usuarios usando Hibernate y PostgreSQL.
	
	% Educación
	\section*{Educación}
	\textbf{Ingeniería de Software} \hfill 2019 -- 2023 \\
	Politécnico Grancolombiano
	
	\textbf{Licenciatura en Matemáticas y Física} \hfill 2006 -- 2009 \\
	Universidad de Antioquia
	
	\textbf{Física (6 semestres)} \hfill 2000 -- 2004 \\
	Universidad de Antioquia
	
	% Habilidades
	\section*{Habilidades Técnicas}
	\textbf{Lenguajes:} Python, Java, C/C++, JavaScript, Bash \\
	\textbf{Frameworks:} Django, Flask, FastAPI, Spring Boot, Hibernate \\
	\textbf{Herramientas:} Docker, Git, Linux, Jenkins, Tailwind CSS, Jupyter \\
	\textbf{Bases de datos:} MySQL, PostgreSQL, AWS RDS \\
	\textbf{Cloud:} AWS (EC2, S3, RDS), Google Cloud
	
	% Idiomas
	\section*{Idiomas}
	\textbf{Español:} Nativo \quad \textbf{Inglés:} B2 (Proficiente)
	
	% Intereses
	\section*{Intereses Profesionales}
	Machine Learning, Seguridad en la nube, Automatización, Sistemas distribuidos, Educación tecnológica
	
\end{document}
